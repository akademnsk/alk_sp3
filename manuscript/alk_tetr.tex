	\documentclass[a4paperm]{article}

\usepackage[T2A]{fontenc}
\usepackage[utf8]{inputenc}
\usepackage[russian,english]{babel}
\usepackage{amsmath,amsthm,amssymb}
\usepackage[affil-it]{authblk}
\usepackage{cite}
\usepackage{scrextend}
\usepackage{verbatim}
\usepackage{paralist}
\usepackage[mediumspace,mediumqspace,Grey,squaren]{SIunits}
\addtokomafont{labelinglabel}{\sffamily}
\usepackage{amsmath}
\usepackage{graphicx}
 \usepackage[usenames, dvipsnames]{color}

\usepackage{SIunits}
\usepackage{miller}
\usepackage[version=3]{mhchem}

\usepackage{float} %H with figures

\setlength{\parindent}{5ex}

\graphicspath{{figures/}}

\begin{document}

\title{Tetrahedrally coordinated carbon in alkaline and alkaline-earth carbonates}
\author[1,2]{Pavel N. Gavryushkin
   \thanks{Electronic address:\texttt{gavryushkin@igm.nsc.ru}; Corresponding author}}     
\author[1,2]{Bekhtenova Altyna}
\author{Evgenii Alexadrov}
\author[1,2]{Nursultan Sagatov}
\author[3]{Zakhar I. Popov}
\author[1,2]{Konstantin D. Litasov}
\affil[1]{Sobolev Institute of Geology and Mineralogy, Siberian Branch of Russian Academy of Sciences, prosp. acad. Koptyuga 3, 630090 Novosibirsk, Russia}
\affil[2]{Novosibirsk State University, Pirogova 2, Novosibirsk 630090, Russia}
\affil[3]{National University of Science and Technology MISIS, 4 Leninskiy pr., Moscow 119049, Russian Federation}


\maketitle

\begin{abstract}
Structures with tetrahedrally coordinated carbon was predicted for alkaline carbonates. The transition without energetical barrier from triangle to tetrahedron. The comparison of tetrahedrally coordinates structures for alkaline and alkaline-earth carbonates show similarity for behaviour of (K2CO3, Na2CO3) and CaCO3. Topological analysis ...

\end{abstract}


\section*{Introduction}
 

\section{Methods}
Altyna

\section{Results}


\subsection{Li2CO3, Na2CO3, K2O3} Altyna

\subsection{SrCO3, BaCO3} Altyna
Absence of tetrahedral structures in the range 0-200 GPa. 

\section{Discussion}

	\paragraph{CaCO3, Na2CO3, K2CO3 – deformational transition from triangle to tetrahedron} 
It seems unique feature of carbonates that transition from sp2 to sp3 bonded carbon does not have energetic barrier.  \\
For alkaline carbonates transition takes place at higher pressures: for CaCO3 and MgCO3 transition – at 70-80 GPa, for Na2CO3 and K2CO3 – at 130 GPa. However, as in case of CaCO3 finding of the new structures can decrease this pressure. 
	
	\paragraph{Li2CO3 – absence of tetrahedrons}  
In case of MgCO3 transition to chain-like structure occurs at 180 GPa. Ring-like structures contain 6 f.u. in the unit cell. If in case of Li2CO3 ring-like structures are also realised they can be missed to big number of atoms in the unit cell. Assuming that for alkaline carbonates transition to tetrahedrons observed at higher pressures than for alkaline-earth, the transition to chain like structure for Li2CO3 should be higher that 180 GPa. In this case, it could fall out from the studied pressure interval (i.e. it is higher than 200 GPa).

	\paragraph{SrCO3, BaCO3 – absence of tetrahedrons}
Absence of tetrahedral structures for SrCO3 and BaCO3 seems enigmatic. As for Li2CO3, we suggest that for these compounds also there are pyroxene-like structures stable at pressures around 100 GPa, which was not found by USPEX due to their complexity.

	
	\paragraph{Chains and rings of tetrahedrons} 
Chains of tetrahedrons in carbonates and in pyroxenes are topologically the same, but geometrically they are sufficiently different. \\
Na2CO3 – straight chain of tetrahedrons in the same orientations. \\
Silicates – neighbouring tetrahedrons are rotated on 180. \\
CaCO3, MgCO3 – two tetrahedrons are parallel, then two – rotated on 180 degree, then two – parallel. 
\par Despite numerous examples of structures with pyroxene-like chains among silicates and phosphates (around 2000 compiunds, file chains.xlsx)  we did not find the topological equivalent of the whole structure for any of the studied carbonate. This uniqueness is likely attributed to the drastic difference in crystallchemistry of carbonates and silicates. Nearly megabar pressure are necessary to stabilise tetrahedral carbonates, while silicon in tetrahedral coordination is stable at ambient conditions.
\par The same is true about ring-like structures of MgCO3. In ICSD we found 206 compounds (silicates, arsenates, phosphates, file rings.xlsx) with topologically equivalent rings, but no one structure with the whole equivalent connectivity.
{\emph Here should be mentioned, that cation nets of other high-pressure structures of carbonates (CaCO3-VI, aragonite-II) also have no or only a few representatives among other compounds.

	\paragraph{Comparison of pyroxene-like structures in alkaline and alkaline-earth carbonates}
At high pressures both MgCO3 and CaCO3 adopt structure with pyroxene-like chains (MgCO3– Pna21, CaCO3-P21/c-h). Chains of tetrahedrons in these structures geometrically are the same. Cation sublattices also have the same topology (10-coordinated bct) but the overall topologies are different.
\par In case of tetrahedral structures of Na2CO3 and K2CO3 not only the geometries of the chains but also the overall topologies of the same.
\par However, as was mentioned above, chains of tetrahedrons in alkaline and alkaline-earth carbonates are sufficiently different.

	\paragraph{Polymerisation of carbon}
At pressures above 400 GPa we found polymerisation of carbon, i.e formation of C-C bonds. For FeCO3 the chains and for SrCO3 – dumpbells have been found. This phenomena is well known for CaC \cite{li2015_cac}, but for the first time we report it for carbonates.





\section*{Acknowledgements}
We thank the Information Technology Centre of Novosibirsk State University for providing access to the cluster computational resources. The research state assignment project (0330-2016-0006).
		

\bibliographystyle{apalike}
\bibliography{alk_tetr}

\end{document}
